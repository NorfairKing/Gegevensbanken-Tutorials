\documentclass[normaalvormen.tex]{subfiles}
\begin{document}

\section*{Voorwoord}
In deze tekst probeer ik het concept van normaalvormen bij `database design' toe te lichten, maar vooral de lezer voor te bereiden op de examenvraag hieromtrent.
Op het examen van Gegevensbanken staat altijd \'e\'en vraag over normaalvormen. Deze wordt uitgebreid besproken en opgelost met een voorbeeld in deze tekst.
De vragen die worden opgelost, worden eerst generisch opgelost, zodat u over een stappenplan beschikt en pas daarna met een voorbeeld.
Deze tekst focust op het praktische aspect, namelijk het oplossen van het examen, meer dan het theoretische: ``Waarom klopt het?''.

\section*{Voorkennis}
\begin{itemize}
\item Eerste orde formele logica.
\item Basis verzamelingenleer.
\item Pseudo-code kunnen lezen.
\end{itemize}
\end{document}
