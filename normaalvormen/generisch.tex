\documentclass[normaalvormen.tex]{subfiles}
\begin{document}

\chapter{Generisch}
\renewcommand\thesection{V\arabic{section}}
\renewcommand\thesubsection{V\arabic{section}}

\section{Bepaal de sluiting van $X$ onder $F$: $X_{F}^{+}$}
\subsection*{Opgave}
Gegeven zijn een attribuutverzameling $X = \{a_1,a_2,\cdot,a_n\}$ en een verzameling functionele afhankelijkheden $F=  \{f_1,f_2,\cdot,f_n\}$. Bepaal $X_{F}^{+}$
\subsection*{Antwoord}
Ga \'e\'en voor \'e\'en de functionele afhankelijkheden in $F$ af. voeg steeds de rechterkant toe als de linkerkant al in je verzameling zit. Herhaal dit tot $X_{F}^{+}$ niet meer verandert.
\begin{mdframed}
\begin{algorithm}[H]
$X_{F}^{+} \leftarrow X$\\
\Repeat{$X_{F}^{+} = OLD$}{
	$OLD \leftarrow X_{F}^{+}$\\
	\ForEach{$f=Y\rightarrow Z \in F$}{
		\If{$y\subseteq X_{F}^{+}$}{
			$X_{F}^{+} = X_{F}^{+} \cup Z$	
		}
	}
}
\end{algorithm}
\end{mdframed}


\section{Bespreek de equivalentie van $E$ en $F$.}
\subsection*{Opgave}
Gegeven zijn twee verzamelingen functionele afhankelijkheden $E$ en $F$. Bereken of $E$ $F$ overdekt, Bereken of $F$ $E$ overdekt en concludeer of $E$ en $F$ equivalent zijn.
\subsection*{Antwoord}
Ga simpelweg alle functionele afhankelijkheden $X\rightarrow Y \in E$ af en controleer of $Y \subseteq X_{F}^{+}$. Controleer het daarna andersom.
\begin{mdframed}
\begin{algorithm}[H]
\ForEach{$e = X\rightarrow Y \in E$}{
	If{ $Y \not\subseteq X_{F}^{+}$}{
		\Return{False}
	}
}
\ForEach{$f = X\rightarrow Y \in F$}{
	If{ $Y \not\subseteq X_{E}^{+}$}{
		\Return{False}
	}
}
\Return{True}
\end{algorithm}
\end{mdframed}

\section{Bereken een minimale overdekking $G$ van $F$.}
\subsection*{Opgave}
Gegeven een verzameling functionele afhankelijkheden $F= \{f_1,f_2,\cdots,f_n$. Bepaal een minimale overdekking $G$ van $F$.
\subsection*{Antwoord}
Ontdubbel eerst de rechterkant van elke functionele afhankelijkheid in $F$. Ontdubbel daarna zoveel mogelijk de linkerkant van elke functionele afhankelijkheid. Let op, dit mag alleen als er dan geen informatie verloren gaat. Elimineer ten slotte alle redundante functionele afhankelijkheden.

\begin{mdframed}
\begin{algorithm}[H]
$G = \emptyset$\\
\tcc{Ontdubbel de rechterkanten.}
\ForEach{$f = X\rightarrow A_1A_2\cdots A_n \ in F$}{
	$G \leftarrow G \cup X\rightarrow A_1 \cup X\rightarrow A_2 \cup \cdots \cup X\rightarrow A_n$
}

\tcc{Ontdubbel zoveel mogelijk de linkerkanten.}
\ForEach{$g = X_1 X_2 \cdots X_n \rightarrow A \in G$}{
	\For{$i = 1 .. n$}{
		$X' \leftarrow X_1 X_2\cdots X_{i-1} X_{i+1}\cdots X_{n}$\\
		$G' \leftarrow G \setminus \{X \rightarrow A\} \cup \{X' \rightarrow  A\}$\\
		\tcc{Enkel wanneer er geen informatie verloren gaat!}
		\If{$X_i \in X_{G'}'^{+}$}{
			$G \leftarrow G'$
		}
	}
}
\tcc{Elimineer redundante functionele afhankelijkheden.}
\ForEach{$g = X\rightarrow A \in G$}{
	$G' \leftarrow G\setminus \{X\rightarrow A\}$\\
	\If{$A \in X_{G'}^{+}$}{
		$G\leftarrow G'$	
	}
}
\end{algorithm}
\end{mdframed}

\section{Bepaal een sleutel voor deze relatie}
\subsection*{Opgave}
Gegeven een relatie $R$ in tabelvorm. Bepaal een sleutel voor $R$.
Geef eventueel de mogelijke supersleutels.
\begin{figure}
\centering
\[
\begin{array}{c|c|c|c}
U_1 & U_2 & \cdots & U_n\\\hline
u_{11} & u_{12} & \cdots & u_{1n}\\
u_{21} & u_{22} & \cdots & u_{2n}\\
\vdots & \vdots & \ddots & \vdots\\
u_{m1} & u_{m2} & \cdots & u_{mn}
\end{array}
\]
\end{figure}

\subsection*{Antwoord}
Bekijk de tabel en probeer functionele afhankelijkheden te zien. Een supersleutel heeft enkel unieke tupels. Zoek uit de verzameling van supersleutels die met het kleinste aantal elementen.

\section{In welke normaalvorm staat dit relatieschema?}
\subsection*{Opgave}
Gegeven een relatieschema $S_{R} = (U,F)$. Bepaal in welke normaalvorm deze relatie staat. Uiteraard word er de hoogst mogelijk normaalvorm bedoeld.
\subsection*{Antwoord}
Ga \'e\'en voor \'e\'en de normaalvormen af. Zodra het relatieschema niet voldoet aan \'e\'en van de voorwaarden stopt u. Noem deze iterate de $(i+1)$-ste. Het relatieschema $S_{R}$ staat dan in $i$-de normaalvorm.

\section{Normaliseer zo ver mogelijk.}
\subsection*{Opgave}
Gegeven een relatieschema $S_{R} = (U,F)$. Normaliseer $S_{R}$ zo ver mogelijk.
\subsection*{Antwoord}
Begin bij de eerste normaalvorm. Breng het relatieschema subsequent  in een volgende normaalvorm indien mogelijk.

\subsubsection*{NF0 $\rightarrow$ NF1}
Omzetten naar een relatieschema in eerste normaalvorm kan altijd.
\begin{enumerate}
\item Splits elk samengesteld attribuut op in meerdere attributen
\item Voorzie voor elk meerwaardig attribuut ofwel meerdere tupels, of wel een nieuwe relatie met dezelfde sleutel.
\end{enumerate}

\subsubsection*{NF1 $\rightarrow$ NF2}
Omzetten naar een relatieschema in tweede normaalvorm kan ook altijd. Voor elk attribuut $A$ dat partieel functioneel afhankelijk is van een kandidaatsleutel $K$
\begin{enumerate}
\item Zoek een subset $K'$ van $K$ waarvan $A$ volledig functioneel afhankelijk is.
\item Verwijder $A$ uit $U$
\item Maak een nieuw relatieschema $S'_{R_{2}}(U',F)$.
\[
U' = K'\cup A
\] 
\end{enumerate}

\subsubsection*{NF2 $\rightarrow$ NF3}
Omzetten naar een relatieschema in derde normaalvorm kan nog steeds altijd. Voor elk niet-sleutelattribuut $A$ dat transitief functioneel afhankelijk is van een kandidaatsleutel via $Z$
\begin{enumerate}
\item Elimineer $A$ uit $U$
\item Maak een nieuw relatieschema $S_{R_{3}}(U',F)$.
\[
U' = Z \cup A
\]
\end{enumerate}

\subsubsection*{NF3 $\rightarrow$ BCNF}
Omzetten naar een relatieschema in BCNF is niet altijd mogelijk zonder extra problemen te veroorzaken.

\end{document}
