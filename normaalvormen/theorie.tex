\documentclass[normaalvormen.tex]{subfiles}
\begin{document}
\chapter{Theorie}
\section{Informele richtlijnen}
\begin{enumerate}
\item Ontwerp een relatieschema zo dat zijn betekenis gemakkelijk verklaard kan worden.

\item Ontwerp een relatieschema zo dat redundantie vermeden wordt en geen toevoeg-, weglaat- of wijziging-anomalie\"en kunnen voorkomen.

\item Vermijd zoveel mogelijk attributen waarvan de waarden nul kunnen zijn.

\item Ontwerp relatieschema's zo dat ze na een equi-join op attributen die primaire of verwijssleutels zijn, geen onechte tupels opleveren.
\end{enumerate}


\section{Functionele afhankelijkheden (functional dependencies)}
\begin{de}
Zij  $X$ en $Y$ attributenverzamelingen. $Y$ is \textbf{functioneel afhankelijk} $X$ als vanuit de waarden van $X$ de waarden van $Y$ deterministisch bepaald kunnen worden.
\[
X\rightarrow Y
\]
In mensentaal:
Stel dat we de $X$ van een tupel kennen, dan kunnen we \emph{met zekerheid} de juiste bijhorende $Y$ vinden.
\end{de}

\begin{de}
Een functionele afhankelijkheid $X\rightarrow Y$ is \textbf{partieel} als er een kleinere deelverzameling $Z$ bestaat van $X$ ($Z\subset X$) zodat Y functioneel afhankelijk is van $Z$.
\[
X \rightarrow Y \text{ maar } \exists Z \subset X \text{ en } Z \rightarrow Y
\]
In mensentaal:
$X$ is meer dan genoeg om $Y$ te bepalen.
\end{de}

\begin{de}
Een functionele afhankelijkheid noemen we \textbf{volledig} wanneer ze niet partieel is.
\[
X \rightarrow Y \text{ en } \not\exists Z \subset X \text{ en } Z \rightarrow Y
\]
In mensentaal:
$X$ is genoeg om $Y$ te bepalen, maar wel maar net genoeg.
\end{de}

\begin{de}
Een functionele afhankelijkheid $X\rightarrow Y$ is \textbf{triviaal} wanneer $Y$ een deel is van $X$.
\[
Y \subseteq X
\]
In mensentaal:
Elk deel van je kennis behoort ook tot je kennis.
\end{de}

\begin{de}
De \textbf{sluiting van een verzameling van attributen} $X$ onder een verzameling functionele afhankelijkheden $X_{F}^+$ is de verzameling van alle attribuutverzamelingen die functioneel afhankelijk zijn van $X$ (al dan niet door andere afhankelijkheden in $F$)
\[
X_{F}^{+} = \{\ Y\ |\ X\rightarrow Y\ \}
\]
In mensentaal:
Alle $Y$ die je \emph{uiteindelijk} te weten kan komen als je $X$ weet (in $F$).
\end{de}

\begin{de}
Een verzameling functionele afhankelijkheden $E$ \textbf{overdekt} een andere verzameling functionele afhankelijkheden $F$ als voor elke $e = X\rightarrow Y$ geldt dat $Y \subseteq X_{F}^{+}$
\[
\forall (X \rightarrow Y) \in E: Y \subseteq X_{F}^{+}
\]
In mensentaal:
Alles dat je te weten kan komen in $E$ kan je ook te weten komen in $F$.
\end{de}

\begin{de}
We spreken van \textbf{equivaltente verzamelingen} $E$ en $F$ van \textbf{functionele afhankelijkheden} als zowel $E$ $F$ overdekt, als $F$ $E$ overdekt.\\
In mensentaal: In $F$ weet je evenveel als in $E$.
\end{de}

\begin{de}
Een verzameling functionele afhankelijkheden $F$ is minimaal als en slechts als er geen equivalente verzameling $G$ te vormen valt door  ...
\begin{itemize}
\item een afhankelijkheid uit $F$ te verwijderen.
\item een attribuut uit de rechterkant van een afhankelijkheid uit $F$ te verwijderen.
\item een attribuut uit de linkerkant van een afhankelijkheid uit $F$ te verwijderen.
\end{itemize}
In mensentaal: Er zit geen overbodige kennis in $F$
\end{de}

\begin{de}
Een functionele afhankelijkheid $X\rightarrow Y$ in een relatieschema $S_{R} = (U,F)$ is \textbf{transitief} als en slechts als er een $Z$ bestaat zodat aan volgende voorwaarden voldaan is.
\begin{itemize}
\item $Z$ is volledig en niet-triviaal afhankelijk van $X$
\item $Z$ is geen deelverzameling van een kandidaatsleutel voor $R$
\item $Y$ is niet-triviaal functioneel afhankelijk van $Z$.
\end{itemize}
\emph{In mensentaal: $X\rightarrow Y$ is een transitieve functionele afhankelijheid als er nog een $Z$ tussen past. ($X\rightarrow Z\rightarrow Y$}
\end{de}

\subsection{Afleidingsregels}
\begin{reg} \textbf{Reflexiviteit}
\[
Y \subseteq X \Rightarrow X \rightarrow Y
\]
\end{reg}
\begin{reg} \textbf{Uitbreiding}
\[
\{X \rightarrow Y\} \models XZ \rightarrow YZ
\]
\end{reg}
\begin{reg} \textbf{Transitiviteit}
\[
\{X \rightarrow Y, Y \rightarrow Z\} \models X \rightarrow Z
\]
\end{reg} 
\begin{reg}\textbf{Decompositie}
\[
\{X \rightarrow YZ\} \models X \rightarrow Y
\]
\end{reg}
\begin{reg} \textbf{Vereniging}
\[
\{X\rightarrow Y, X\rightarrow Z \} \models X\rightarrow YZ
\]
\end{reg}
\begin{reg} \textbf{Pseudo-Transitiviteit}
\[
\{ X\rightarrow Y, WY \rightarrow Z \} \models WX\rightarrow Z
\]
\end{reg}

\section{Sleutels}
\begin{de}
Een verzameling attributen $K$ is een \textbf{supersleutel} voor een relatie $R$ met schema $S_{R} = (U,F)$ als en slechts als $K$ elk attribuut in $U$ determineert.
\[
\forall u \in U:  K\rightarrow u \text{ of ook } K_{F}^{+} = U
\]
In mensentaal:
Uit een supersleutel kan je \emph{uiteindelijk} al de rest afleiden.
\end{de}

\begin{de}
Een verzameling attributen $K$ is een \textbf{(kandidaat)sleutel}  voor een relatie $R$ met schema $S_{R} = (U,F)$ als en slechts als $K$ een supersleutel is en er geen kleinere deelverzameling van $K$ een supersleutel is voor $R$.\\
In mensentaal: Een (kandidaat)sleutel is \'e\'en van de kleinste verzamelingen waaruit al de rest kan afgeleid worden.
\end{de}

\begin{de}
Een attribuut $A$ is een sleutelattribuut voor een relatie $R$ met schema $S_{R} = (U,F)$ als en slecht als er een sleutel $K$ voor $R$ bestaat waar $A$ in zit.
In mensentaal: Een sleutelattribuut is een attribuut in een sleutel.
\end{de}

\section{Normaalvormen}
\begin{de}
Elk relatieschema is in \textbf{nulde normaalvorm (NF0)}. Er zijn geen voorwaarden opgelegd aan de attributen of functionele afhankelijkheden.
\end{de}

\begin{de}
Een relatieschema $S_{R} = (U,F)$ is in \textbf{eerste normaalvorm (NF1)} als en slechts als het domein van elk attribuut in $U$ enkelvoudig is. Er zijn dus geen meervoudige attributen in $U$.
\end{de}

\begin{de}
Een relatieschema $S_{R} = (U,F)$ is in \textbf{tweede normaalvorm (NF2)} als en slechts als voor elk niet-sleutelattribuut $A \in U$ geldt dat $A$ volledig functioneel afhankgelijk is van \emph{een} kandidaatsleutel van $R$.\\
In mensentaal: Elk niet-sleutel-attribuut is \emph{enkel} volledig, \emph{eventueel transitief}, afhankelijk van (een) supersleutel(s). (Je kan het niet determineren met minder dan een supersleutel)
\end{de}

\begin{de}
Een relatieschema $S_{R} = (U,F)$ is in \textbf{derde normaalvorm (NF3)} als en slechts als het volgende geldt voor elke niet-triviale functionele afhankelijkheid $X\rightarrow A$.
\[
A \text{ is een sleutel-attribuut van } R \text{ of } X \text{ is een supersleutel voor R}
\]
In mensentaal: Elk niet-sleutel-attribuut is \textbf{enkel} volledig, \emph{niet-transitief<}, afhankelijk van \emph{elke} supersleutel.
\end{de}

\begin{de}
Een relatieschema $S_{R} = (U,F)$ is in \textbf{Boyce-Codd normaalvorm (BCNF)} als en slechts als geen enkel sleutelattribuut van $U$ partieel of transitief functioneel afhankelijk is van een kandidaatsleutel van $R$.\\
In mensentaal: Supersleutels zijn  noch rechtstreeks, noch transitief afhankelijk van niet-sleutel- attributen.
\end{de}

Let op: De definities zijn inderdaad incrementeel. Ze bouwen voort op elkaar, ookal is dat niet expliciet aangegeven in de definitie.

\subsection{Voorbeelden}
\subsubsection*{2NF}
\textbf{Wel} in tweede normaalvorm:
\begin{itemize}
\item
\[
(U,F) = (\{\underline{A}, \underline{B}, C\},\{AB\rightarrow C\})
\]
\item
\[
(U,F) = (\{\underline{A},B,C\},\{A\rightarrow BC, B\rightarrow C\})
\]
\item
\[
(U,F) = (\{\underline{A},B,C,D,E\},\{A\rightarrow BCDE, BC\rightarrow ADE, D\rightarrow E\})
\]
\end{itemize}
\textit{Niet} in tweede normaalvorm:
\begin{itemize}
\item
\[
(U,F) = (\{\underline{A}, \underline{B}, C\},\{AB\rightarrow C, A \rightarrow C\})
\]
$C$, een niet-sleutel-attribuut is hier nog partieel afhankelijk van een supersleutel ($AB$).
\item
\[
(U,F) = (\{\underline{A}, \underline{B}, C , D\},\{AB\rightarrow CD, B \rightarrow D\})
\]
$D$, een niet-sleutel-attribuut is hier nog partieel afhankelijk van een supersleutel ($AB$).
\item
\[
(U,F) = (\{\underline{A},B,C,D,E\},\{A\rightarrow BCDE, BC\rightarrow ADE, B\rightarrow D, D\rightarrow E\})
\]
$D$, een niet-sleutel-attribuut is hier nog partieel afhankelijk van een supersleutel ($BC$);
\end{itemize}


\subsubsection*{3NF}

\subsubsection*{BCNF}

\end{document}
