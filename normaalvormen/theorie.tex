\documentclass[normaalvormen.tex]{subfiles}
\begin{document}
\chapter{Theorie}
\section{Informele richtlijnen}
\begin{enumerate}
\item Ontwerp een relatieschema zo dat zijn betekenis gemakkelijk verklaard kan worden.

\item Ontwerp een relatieschema zo dat redundantie vermeden wordt en geen toevoeg-, weglaat- of wijziging-anomalie\"en kunnen voorkomen.

\item Vermijd zoveel mogelijk attributen waarvan de waarden nul kunnen zijn.

\item Ontwerp relatieschema's zo dat ze na een equi-join op attributen die primaire of verwijssleutels zijn, geen onechte tupels opleveren.
\end{enumerate}


\section{Functionele afhankelijkheden (functional dependencies)}
\begin{de}
Zij  $X$ en $Y$ attributenverzamelingen. $Y$ is \textbf{functioneel afhankelijk} $X$ als vanuit de waarden van $X$ de waarden van $Y$ deterministisch bepaald kunnen worden.
\[
X\rightarrow Y
\]
\end{de}
\begin{de}
Een functionele afhankelijkheid $X\rightarrow Y$ is \textbf{partieel} als er een kleinere deelverzameling $Z$ bestaat van $X$ ($Z\subset X$) zodat Y functioneel afhankelijk is van $Z$. ($Z\rightarrow Y$)
\end{de}
\begin{de}
Een functionele afhankelijkheid noemen we \textbf{volledig} wanneer ze niet partieel is.
\end{de}
\begin{de}
Een functionele afhankelijkheid $X\rightarrow Y$ is \textbf{triviaal} wanneer $Y$ een deel is van $X$.
\[
Y \subseteq X
\]
\end{de}
\begin{de}
De \textbf{sluiting van een verzameling van attributen} $X$ onder een verzameling functionele afhankelijkheden $X_{F}^+$ is de verzameling van alle attribuutverzamelingen die functioneel afhankelijk zijn van $X$
\[
X_{F}^{+} = \{\ Y\ |\ X\rightarrow Y\ \}
\]
\end{de}
\begin{de}
Een verzameling functionele afhankelijkheden $E$ \textbf{overdekt} een andere verzameling functionele afhankelijkheden $F$ als voor elke $e = X\rightarrow Y$ geldt dat $Y \subseteq X_{F}^{+}$
\end{de}
\begin{de}
We spreken van \textbf{equivaltente verzamelingen} $E$ en $F$ van \textbf{functionele afhankelijkheden} als zowel $E$ $F$ overdekt, als $F$ $E$ overdekt.
\end{de}
\begin{de}
Een verzameling functionele afhankelijkheden is minimaal als en slechts als er geen equivalente verzameling $G$ te vormen valt door  ...
\begin{itemize}
\item een afhankelijkheid uit $F$.
\item een attribuut uit de rechterkant van een afhankelijkheid uit $F$.
\item een attribuut uit de linkerkant van een afhankelijkheid uit $F$.
\end{itemize}
\end{de}
\begin{de}
Een functionele afhankelijkheid $X\rightarrow Y$ in een relatieschema $S_{R} = (U,F)$ is \textbf{transitief} als en slechts als er een $Z$ bestaat zodat aan volgende voorwaarden voldaan is.
\begin{itemize}
\item $Z$ is volledig en niet-triviaal afhankelijk van $X$
\item $Z$ is geen deelverzameling van een kandidaatsleutel voor $R$
\item $Y$ is niet-triviaal functioneel afhankelijk van $Z$.
\end{itemize}
\emph{In mensentaal: $X\rightarrow Y$ is een transitieve functionele afhankelijheid als er nog een $Z$ tussen past. ($X\rightarrow Z\rightarrow Y$}
\end{de}
\subsection{Afleidingsregels}
\begin{reg} \textbf{Reflexiviteit}
\[
Y \subseteq X \Rightarrow X \rightarrow Y
\]
\end{reg}
\begin{reg} \textbf{Uitbreiding}
\[
\{X \rightarrow Y\} \models XZ \rightarrow YZ
\]
\end{reg}
\begin{reg} \textbf{Transitiviteit}
\[
\{X \rightarrow Y, Y \rightarrow Z\} \models X \rightarrow Z
\]
\end{reg} 
\begin{reg}\textbf{Decompositie}
\[
\{X \rightarrow YZ\} \models X \rightarrow Y
\]
\end{reg}
\begin{reg} \textbf{Vereniging}
\[
\{X\rightarrow Y, X\rightarrow Z \} \models X\rightarrow YZ
\]
\end{reg}
\begin{reg} \textbf{Pseudo-Transitiviteit}
\[
\{ X\rightarrow Y, WY \rightarrow Z \} \models WX\rightarrow Z
\]
\end{reg}

\section{Sleutels}
\begin{de}
Een verzameling attributen $K$ is een \textbf{supersleutel} voor een relatie $R$ met schema $S_{R} = (U,F)$ als en slechts als $K$ elk attribuut in $U$ determineert.
\[
K_{F}^{+} = U
\]
\end{de}
\begin{de}
Een verzameling attributen $K$ is een \textbf{(kandidaat)sleutel}  voor een relatie $R$ met schema $S_{R} = (U,F)$ als en slechts als $K$ een supersleutel is en er geen kleinere deelverzameling van $K$ een supersleutel is voor $R$.
\end{de}
\begin{de}
Een attribuut $A$ is een sleutelattribuut voor een relatie $R$ met schema $S_{R} = (U,F)$ als en slecht als er een sleutel $K$ voor $R$ bestaat waar $A$ in zit.
\end{de}
\section{Normaalvormen}
\begin{de}
Elk relatieschema is in \textbf{nulde normaalvorm (NF0)}. Er zijn geen voorwaarden opgelegd aan de attributen of functionele afhankelijkheden.
\end{de}
\begin{de}
Een relatieschema $S_{R} = (U,F)$ is in \textbf{eerste normaalvorm (NF1)} als en slechts als het domein van elk attribuut in $U$ enkelvoudig is.
\end{de}
\begin{de}
Een relatieschema $S_{R} = (U,F)$ is in \textbf{tweede normaalvorm (NF2)} als en slechts als voor elk niet-sleutelattribuut $A \in U$ geldt dat $A$ partieel functioneel afhankgelijk is van een kandidaatsleutel van $R$.
\emph{In mensentaal: ``Voor elk niet-sleutel-attribuut moet de hele primaire sleutel nodig zijn om het te determineren.''}
\end{de}
\begin{de}
Een relatieschema $S_{R} = (U,F)$ is in \textbf{derde normaalvorm (NF3)} als en slechts als volgende implicatie geldt voor elke niet-triviale functionele afhankelijkheid $X\rightarrow A$.
\[
A \text{ is een niet-sleutel-attribuut } \Rightarrow X \text{ is een supersletel voor R}
\]
\emph{In mensentaal: Niet-sleutel-attributen zijn enkel afhankelijk van supersleutels.}
\end{de}
\begin{de}
Een relatieschema $S_{R} = (U,F)$ is in \textbf{Boyce-Codd normaalvorm (BCNF)} als en slechts als geen enkel sleutelattribuut van $U$ partieel of transitief functioneel afhankelijk is van een kandidaatsleutel van $R$.
\emph{In mensentaal: Supersleutels zijn onafhankelijk.}
\end{de}


\end{document}
