\documentclass[normaalvormen.tex]{subfiles}
\begin{document}
\chapter{Theorie}
\section{Informele richtlijnen}
\begin{enumerate}
\item Ontwerp een relatieschema zo dat zijn betekenis gemakkelijk verklaard kan worden.

\item Ontwerp een relatieschema zo dat redundantie vermeden wordt en geen toevoeg-, weglaat- of wijziging-anomalie\"en kunnen voorkomen.

\item Vermijd zoveel mogelijk attributen waarvan de waarden nul kunnen zijn.

\item Ontwerp relatieschema's zo dat ze na een equi-join op attributen die primaire of verwijssleutels zijn, geen onechte tupels opleveren.
\end{enumerate}


\section{Functionele afhankelijkheden (functional dependencies)}
\begin{de}
Zij  $X$ en $Y$ attributenverzamelingen. $Y$ is functioneel afhankelijk $X$ als vanuit de waarden van $X$ de waarden van $Y$ deterministisch bepaald kunnen worden.
\[
X\rightarrow Y
\]
\end{de}
\begin{de}
De sluiting van een verzameling van attributen $X$ onder een verzameling functionele afhankelijkheden $X_{F}^+$ is de verzameling van alle attribuutverzamelingen die functioneel afhankelijk zijn van $X$
\[
X_{F}^{+} = \{\ Y\ |\ X\rightarrow Y\ \}
\]
\end{de}
\subsection{Afleidingsregels}
\begin{reg} \textbf{Reflexiviteit}
\[
Y \subseteq X \Rightarrow X \rightarrow Y
\]
\end{reg}
\begin{reg} \textbf{Uitbreiding}
\[
\{X \rightarrow Y\} \models XZ \rightarrow YZ
\]
\end{reg}
\begin{reg} \textbf{Transitiviteit}
\[
\{X \rightarrow Y, Y \rightarrow Z\} \models X \rightarrow Z
\]
\end{reg} 
\begin{reg}\textbf{Decompositie}
\[
\{X \rightarrow YZ\} \models X \rightarrow Y
\]
\end{reg}
\begin{reg} \textbf{Vereniging}
\[
\{X\rightarrow Y, X\rightarrow Z \} \models X\rightarrow YZ
\]
\end{reg}
\begin{reg} \textbf{Pseudo-Transitiviteit}
\[
\{ X\rightarrow Y, WY \rightarrow Z \} \models WX\rightarrow Z
\]
\end{reg}


\section{Sleutels}

\section{Nulde normaalvorm: NF0}
\section{Eerste normaalvorm: NF1}
\section{Tweede normaalvorm: NF2}
\section{Derde normaalvorm: NF3}
\section{Boyce-Codd normaalvorm: BNCF}
\section{Vierde normaalvorm: NF4}
\section{Vijfde normaalvorm: NF5}

\end{document}
