\documentclass[10pt,a4paper,oneside]{report}
\usepackage[latin1]{inputenc}
\usepackage[dutch]{babel}


% Voor subfiles
\usepackage{subfiles}

% Voor algoritmes
\usepackage{algorithm2e}

% Voor todo's
\usepackage{todonotes}

% Voor wiskunde
\usepackage{amsmath}
\usepackage{amsfonts}
\usepackage{amssymb}
\usepackage{amsthm}

% Voor urls
\usepackage{hyperref}

% svg
\usepackage[clean,pdf]{svg}
\setsvg{svgpath = illustraties/}
\setsvg{inkscape = inkscape -z -D}

% Om het totaal aantal pagina's te tellen
\usepackage{lastpage}
\usepackage{afterpage}

% Voor tekeningen
\usepackage{tikz}
\usetikzlibrary{decorations}
\usetikzlibrary{calc}
\usetikzlibrary{arrows.meta}

% Nog tekeningen
\usepackage{pgfplots}

% SVG tekeningen
\usepackage{svg}

% Om figuren op de juiste plaats te krijgen
\usepackage{float}

% Hyperref zorgt voor clickable stuff.
\usepackage{hyperref}

% Voor frames
\usepackage{mdframed}

% Om de marges aan te passen
\usepackage[left=2cm,right=2cm,top=2cm,bottom=2cm]{geometry}

% Voor headers en footers
\usepackage{fancyhdr}
% fancy verbatim
\usepackage{fancyvrb}
% program listings
\usepackage{listings}
% clickable TOC
\usepackage{hyperref}
\hypersetup{
    colorlinks,
    citecolor=black,
    filecolor=black,
    linkcolor=black,
    urlcolor=black
}

\theoremstyle{plain}
\newtheorem{thm}{Theorem}[chapter] %Reset counter elk hoofdstuk

\theoremstyle{definition}
\newmdtheoremenv{de}[thm]{Definitie} % Definitie met frame
\newtheorem{ei}[thm]{Eigenschap}
\newtheorem{st}[thm]{Stelling}
\newtheorem{gev}[thm]{Gevolg}
\newtheorem{reg}[thm]{Regel}

\begin{document}
\begin{titlepage}
\thispagestyle{empty}
\newcommand{\HRule}{\rule{\linewidth}{0.5mm}}
\center
\textsc{\LARGE KU Leuven}\\[1.5cm]
\vfill


{ \Huge \bfseries Tutorial: Hashing}\\[0.4cm]
% \HRule \\[1.5cm]
\textsc{\large Gegevensbanken [H01O9A]}\\[0.5cm]

\vspace{5cm}

\begin{Large}
Gestart: 3 juni 2014\\
Cecompileerd: \today\\
\end{Large}
\vspace{5cm}

\begin{minipage}{0.4\textwidth}
\begin{flushleft} \large
\emph{Auteur:}\\
Tom Sydney \textsc{Kerckhove}
\end{flushleft}
\end{minipage}
~
\begin{minipage}{0.4\textwidth}
\begin{flushright} \large
\emph{Professor:} \\
Bettina \textsc{Berendt}\\
\end{flushright}
\end{minipage}\\[4cm]

\vfill

\end{titlepage}

\subfile{voorwoord}
\tableofcontents
\pagebreak


%\nocite{tmi}
%\bibliographystyle{plain}
%\bibliography{computergesteund_ontwerp_van_curven_en_oppervlakken}


\section*{Voorkennis}
\begin{itemize}
\item Eerste orde formele logica.
\end{itemize}

\subfile{theorie}

\chapter{Generisch}
\renewcommand\thesection{Vraag \arabic{section}}
\renewcommand\thesubsection{V \arabic{section}}
\section{Bepaal de sluiting van $X$ onder $F$: $X_{F}^{+}$}
\subsection{Opgave}
Gegeven zijn een attribuutverzameling $X = \{a_1,a_2,\cdot,a_n\}$ en een verzameling functionele afhankelijkheden $F=  \{f_1,f_2,\cdot,f_n\}$.
\subsection{Antwoord}

\chapter{Voorbeeld}
\section{Bepaal de sluiting van $X$ onder $F$: $X_{F}^{+}$}
\subsection{Opgave}
\[
X1 = \{A\}
X2 = \{C\}
X3 = \{AC\}
\]
\[
F = 
\{
  A\rightarrow B
, C\rightarrow DE
, AC \rightarrow F
\}
\]

\subsection{Antwoord}
\[
\]


\end{document}