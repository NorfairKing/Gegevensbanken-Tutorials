\documentclass[indexstructuren.tex]{subfiles}
\begin{document}

\chapter{Generisch}
\renewcommand\thesection{V\arabic{section}}
\renewcommand\thesubsection{V\arabic{section}}

\section{Toeganssnelheid}
\subsection*{Opgave}
Gegeven een disk met een blok groote $B$. Een wijzer naar een blok heeft groote $P_b$ en een wijzer naar een record heeft groote $P_r$. Een bestand heeft $r$ records van vaste grootte $R$. Elk record heeft groote $i$ velden met respectievelijke grootte $I_i$. Een deletion marker heeft groote $I_{del}$.
\begin{itemize}
\item Bereken de record grootte $R$.
\item Bereken de blocking factor $bfr$.
\item \'e\'en van de zes volgende vragen.
\begin{itemize}
\item Gegeven dat het bestand ge\"ordend is op veld $j$ (met unieke waarden) en we een primaire index willen construeren op dat veld.
\begin{itemize}
\item Bereken de index blocking factor $bfr_i$ en de fan-out $fo$.
\item Bereken het totaal records op het eerste indexeringsniveau $r_{i_1}$ en het aantal blokken op het eerste indexeringsniveau $b_{i_1}$.
\item Bereken het totaal aantal niveaus $h$ nodig als we een meervoudige index zouden maken. 
\item Bereken het totaal aantal records $r_{i_{tot}}$ en het totaal aantal blokken $b_{i_{tot}}$ nodig voor de hele meervoudige index.
\item Bereken het totaal aantal blok toegangen nodig om een record te vinden.
\end{itemize}

\item
Gegeven dat het bestand niet geordend is op veld $j$ (met unieke waarden) en we een secundaire index willen construeren op dat veld.
\begin{itemize}
\item Bereken de index blocking factor $bfr_i$ en de fan-out $fo$.
\item Bereken het totaal records op het eerste indexeringsniveau $r_{i_1}$ en het aantal blokken op het eerste indexeringsniveau $b_{i_1}$.
\item Bereken het totaal aantal niveaus $h$ nodig als we een meervoudige index zouden maken. 
\item Bereken het totaal aantal records $r_{i_{tot}}$ en het totaal aantal blokken $b_{i_{tot}}$ nodig voor de hele meervoudige index.
\item Bereken het totaal aantal blok toegangen $t$ nodig om een record te vinden.
\end{itemize}

\item 
Gegeven dat het bestand niet geordend is op het veld $j$, en we de derde optie gebruiken voor een secundaire index. Er is dus een extra niveau van indirectie voor record wijzers. Gegeven dat er $u_j$ unieke waarden van het $j$-de veld zijn.
\begin{itemize}
\item Bereken de index blocking factor en de fan-out $fo$.
\item Bereken het aantal blokken $b_0$ en het aantal records $r_0$ nodig voor het extra niveau van indirectie.
\item Bereken het totaal records op het eerste indexeringsniveau $r_{i_1}$ en het aantal blokken op het eerste indexeringsniveau $b_{i_1}$.
\item Bereken het totaal aantal niveaus $h$ nodig als we een meervoudige index zouden maken. 
\item Bereken het totaal aantal records $r_{i_{tot}}$ en het totaal aantal blokken $b_{i_{tot}}$ nodig voor de hele meervoudige index.
\item Bereken het totaal aantal blok toegangen nodig om een record te vinden.
\end{itemize}

\item
Gegeven dat een bestand is geordend op het veld $j$, en we een clusterende index willen construeren op dat veld. De index gebruikt blok ankers. Gegeven dat er $u_j$ unieke waarden van het $j$-de veld zijn.
\begin{itemize}
\item Bereken de index blocking factor $bfr_i$ en de fan-out $fo$.
\item Bereken het totaal records op het eerste indexeringsniveau $r_{i_1}$ en het aantal blokken op het eerste indexeringsniveau $b_{i_1}$.
\item Bereken het totaal aantal niveaus $h$ nodig als we een meervoudige index zouden maken. 
\item Bereken het totaal aantal records $r_{i_{tot}}$ en het totaal aantal blokken $b_{i_{tot}}$ nodig voor de hele meervoudige index.
\item Bereken het totaal aantal blok toegangen nodig om een record te vinden.
\end{itemize}

\item
Gegeven dat een bestand is geordend op het veld $j$ met unieke waarden, en we een B (of B+) boom willen construeren op dat veld.
\begin{itemize}
\item Bereken de ordes $p_{knoop}$ en $p_{blad}$ voor een B+ boom en de orde $p'_{knoop}$ voor een B boom.
\item Bereken het aantal bladeren nodig $b_l$ voor een B+ boom en $b'_l$ voor een B boom als de B(+) boom $L$ vol zit.
\item Bereken het aantal niveaus $h$ nodig als alle knopen $L$ vol zitten.
\item Bereken het totaal aantal blokken $b^{*}$ en $b'^{*}$ nodig voor de B+ en B boom.
\end{itemize}

\end{itemize}
\end{itemize}
\subsection*{Antwoord}
We berekenen  \'e\'en voor \'e\'en alle gevraagde waarden.
\begin{itemize}
\item Record grootte $R$
\[
R = \sum_{k=1}^{i}I_k
\]
\item De blocking factor $bfr$
\[
bfr = \left\lfloor \frac{B}{R} \right\rfloor
\]
\item Index record grootte $R_i$
\[
R_{i} = I_{j} + P
\]
\item De index blocking factor $bfr_0$ voor het extra niveau van indirectie bij een secundaire index op een niet-uniek-veld en het aantal blokken voor dat niveau $b_{R}$.
\[
bfr_0 = \left\lfloor \frac{B}{P_R} \right\rfloor \text{ en } b_{R} = \left\lceil \frac{b}{bfr_R} \right\rceil 
\]
\item De index blocking factor $bfr_i = fo$
\[
bfr_i = fo = \left\lfloor \frac{B}{R_i} \right\rfloor
\]
\item Het aantal records op het eerste index niveau $b_{i_{1}}$
\[
b_{i_{1}} = \left\lceil \frac{r_{i_{1}}}{bfr_i}\right\rceil
\]
\item Hoogte van een meervoudige index $h$
\[
\lceil \log_{fo}(b) \rceil
\]
\item Het totaal aantal records in de index
\[
r_{i_{tot}} = \sum_{k=1}^{h}\left\lceil\frac{b_{i_k}}{bfr_i}\right\rceil
\]
\item Het totaal aantal blokken in de index
\[
b_{i_{tot}} = \sum_{k=1}^h\left\lceil \frac{r_{i_{k}}}{bfr_{i_k}} \right\rceil
\]
\item Het totaal aantal blok toegangen nodig bij \'e\'en niveau van indexering.
\[
t_{one} = \lceil \log_{2}b_{i_{1}} \rceil
\]
\item Het totaal aantal blok toegangen nodig bij een meervoudige indexering.
\[
t_{multi} = t_{indexeringsniveaus} + t_{indirectie} + t_{datatoegang}
\]
\item Ordes van B bomen
\[
p_{knoop} P + (p_{knoop} - 1)I_{j} \le B
\]
\[
p_{blad}(P_{R} + I_{j}) \le B
\]
\[
p'_{knoop} P + (p'_{knoop}-1)P_{R} + (p-1)I_{j} \le B
\]
\item Samenvatting van de verschillen
\begin{figure}[H]
\centering
\begin{tabular}{r|ccccccccc}
& $r_{i_{1}}$ & $t_{multi}$\\
\hline
primaire index & $b$ & $h+1$\\
secundaire index (1) & $r$ & $h+1$\\
secundaire index (2) & $u_j$ & $h+2$\\
clusterende index & $u_j$ & $h+\left\lceil\frac{u_j}{bfr}\right\rceil$\\
B+ boom & \\
B boom & \\
\end{tabular}
\end{figure}
\end{itemize}

\end{document}
