\documentclass[hashing.tex]{subfiles}
\begin{document}

\section*{Voorwoord}
De theorie rond hashing is niet zo goed uitgelegd, noch in het boek, noch in de slides. Ik heb ze daarom hier opnieuw uitgeschreven tot in  (waarschijnlijk iets \emph{te}) groot detail.
Hashing wordt bovendien elk jaar gevraagd op het examen. De vraag is vaak gelijkaardig aan die van andere jaren. Ik heb specifiek zo een examenvraag opgelost als voorbeeld.

\section*{Voorkennis}
\begin{itemize}
\item Basis analyse.
\item Modulo rekenen.
\item Het concept van records en sleutels.
\item Pseudo code kunnen lezen.
\end{itemize}


\end{document}
