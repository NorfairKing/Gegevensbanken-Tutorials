 \documentclass[indexstructuren.tex]{subfiles}
\begin{document}

\chapter{Voorbeeld}
\renewcommand\thesection{V\arabic{section}}
\renewcommand\thesubsection{V\arabic{section}}
\section{Toeganssnelheid}
\subsection*{Opgave}
Gegeven een disk met blok grootte $B=1024$ bytes Een wijzer naar een blok heeft grootte $P_{b} = 6$ bytes en een wijzer naar een record heeft grootte $P_{r} = 8$ bytes. Een bestand heeft $100000$ records van vaste grootte $R$. Een record bestaat uit de volgende velden. Er wordt \'e\'en byte gebruikt voor een deletion marker. Het bestand is geordend op $A$. $A$ en $B$ hebben unieke waarden, $C$ niet noodzakelijk.
\begin{figure}[H]
\centering
\begin{tabular}{c|c}
Veld & grootte\\\hline
$A$ & $8$ bytes\\
$B$ & $16$ bytes\\
$C$ & $64$ bytes\\
\end{tabular}
\end{figure}
\subsection*{Antwoord}
De record grootte $R$
\[
R = 8 + 16 + 64 = 88
\]
De blocking factor $bfr$.
\[
bfr = \left\lfloor\frac{1024}{88}\right\rfloor = 11
\]
Aantal blokken voor het databastand $b$
\[
b = \left\lceil \frac{100000}{11} \right\rceil  = 9091
\]
\begin{itemize}
\item Primaire index op $A$.
\begin{itemize}
\item De index record groottes $R_i$ en $R'_i$
\[
R_i = I_A + P_r = 8 + 8 = 16 \text{ en } R'_i = I_{A} + P_b = 8 + 6 = 14
\]
\item De index blocking factor $bfr_i$ en $bfr'_i$
\[
bfr_{i} = \left\lfloor\frac{1024}{16}\right\rfloor = 64 \text{ en }bfr'_{i} = \left\lfloor\frac{1024}{14}\right\rfloor = 73
\]
\item Totaal aantal records op het eerste indexeringsniveau $r_{i_{1}}$ en het totaal aantal blokken op dat niveau $b_{i_{1}}$.
\[
r_{i_{1}} = b = 9091 \text{ en } b_{i_{1}} = \left\lceil\frac{9091}{64} \right\rceil = 124
\]
\item Het totaal aantal niveaus voor een meervoudige index $h$
\[
h = \left\lceil \log_{73}b_{i_{1}} \right\rceil +1= 3
\]
Inderdaad $b_{i_{2}} = \left\lceil \frac{124}{73} \right\rceil = 2$ en $b_{i_{3}} = 1$.
\item Het totaal aantal index records $r_{i_{tot}}$ en blokken $b_{i_{tot}}$.
\[
r_{i_{tot}} = 9091 + 124 + 2 = 9217 \text{ en } b_{i_{tot}} = 124 + 2 +1
\]
\item Het totaal aantal toegangen $t$ nodig om een record te vinden.
\[
t_{one} = \lceil \log_{2} 144\rceil + 1 = 8+1 = 9 \text{ en } t_{multi} = 3 + 1 = 4 
\]
\end{itemize}
\item Secundaire index (1) op $B$.
TODO
\item Secundaire index (2) op $C$.
TODO
\item Clusterende index op $C$. (Stel dat het bestand geordend zou zijn op $C$)
TODO
\item B+ boom op $B$.
TODO
\item B boom op $B$.
TODO
\end{itemize}

\end{document}
