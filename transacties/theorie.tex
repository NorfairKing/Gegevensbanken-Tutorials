\documentclass[transacties.tex]{subfiles}
\begin{document}
\chapter{Theorie}

\section{Transacties (Transactions)}
\begin{de}
Een transactie $T$ is de uitvoering van een programma dat de gegevensbank raadpleegt of haar inhoud wijzigt.
\end{de}

\begin{de}
We spreken van een \textbf{verloren aanpassing} (lost modification) wanneer een wijziging door een transactie $T_1$ teniet gedaan wordt door een andere transactie $T_2$.
\end{de}

\begin{de}
Een \textbf{foute lezing door tijdelijke aanpassing} (dirty read) vindt plaats wanneer een transactie $T_1$ een waarde gebruikt (het resultaat van een andere transactie $T_2$) die ondertussen niet meer geldig is.
\end{de}

\begin{de}
Een \textbf{niet herhaalbare lezing} (nonrepeatable read) vindt plaats wanneer een waarde gelezen wordt en daarna wordt aangepast. zodat de originele waarde niet opnieuw te lezen valt.
\end{de}

\begin{de}
De \textbf{ideale transactie} voldoet aan vier belangrijke eigenschappen. (ACID)
\begin{enumerate}
\item \textbf{Ondeelbaarheid (atomicity)}
\item \textbf{Consistentie behoudbaarheid (consistency presercation)}\\
\item \textbf{Isolatie (Isolation)}
\item \textbf{Duurzaamheid (Durability)}
\end{enumerate}
\end{de}

\section{Transactieroosters (Schedules}
\begin{de}
Een \textbf{transactierooster} (schedule) is een tabel van operaties, in chronologische volgorde voor meerdere transacties. In de kolommen staan chronologisch de operaties van \'e\'en transactie.
\end{de}
Voorbeeld:
\[
\begin{array}{c|c}
T_1 & T_2\\
\hline
o_1 & \\
& o_2\\
o_3&\\
o_4&\\
& o_5\\
\end{array}
\]
In deze tekst gaan we ervan uit dat operaties niet tegelijkertijd uitgevoerd worden.


\begin{de}
We noemen twee operaties \textbf{conflicterend} (conflicting) als... :
\begin{itemize}
\item ze bij verschillende transacties horen.
\item ze hetzelfde gegevenselement gebruiken.
\item minstens \'e\'en ervan een schrijfoperatie is.
\end{itemize}
\end{de}

\begin{de}
We noemen een rooster $S$ voor $n$ interacties $T_i$ \textbf{volledig} als... :
\begin{itemize}
\item $S$ alle operaties van alle transacties $T_i$ bevat, alsook een commit- of abort-operatie op het einde van de transactie.
\item elk paar operaties van elke transactie $T_i$ in dezelfde volgorde voorkomt in $S$ als in $T_i$
\item de volgorde van elk paar conflicterende operaties eenduidig vast ligt.
\end{itemize}
\end{de}

\subsection{Herstelbaarheid}
\begin{de}
Een rooster is \textbf{herstelbaar} als en slecht als een transactie die gecommit is nooit meer ongedaan gemaakt moet worden.
Dit houd in dat geen enkele transactie $T$ in $S$ commit voordat alle transacties $T'$ die iets geschreven hebben dat $T$ leest hebben gecommit. 
\end{de}
\begin{figure}[H]
\centering
\begin{tabular}{c|c}
\multicolumn{2}{c}{Herstelbaar}\\
$T_1$&$T_2$\\
\hline
read(A) &\\
& read(A)\\
write(A)&\\
read(B)&\\
write(B)&\\
&write(A)\\
&commit\\
commit&
\end{tabular}
\begin{tabular}{c|c}
\multicolumn{2}{c}{Niet herstelbaar}\\
$T_1$&$T_2$\\
\hline
read(A) &\\
write(A)&\\
& read(A)\\
read(B)&\\
&write(A)\\
&commit\\
abort&
\end{tabular}
\begin{tabular}{c|c}
\multicolumn{2}{c}{Herstelbaar}\\
$T_1$&$T_2$\\
\hline
read(A) &\\
write(A)&\\
& read(A)\\
read(B)&\\
&write(A)\\
write(B)&\\
commit&\\
&commit\\
\end{tabular}
\begin{tabular}{c|c}
\multicolumn{2}{c}{Herstelbaar}\\
$T_1$&$T_2$\\
\hline
read(A) &\\
write(A)&\\
& read(A)\\
read(B)&\\
&write(A)\\
write(B)&\\
abort&\\
&abort\\
\end{tabular}
\end{figure}

\subsection{Cascadeloosheid}
\begin{de}
Een rooster $S$ is \textbf{cascadeloos} wanneer elke transactie $T$ enkel waarden leest die geschreven zijn door transacties die al gecommit zijn.
\end{de}
\begin{figure}[H]
\centering
\begin{tabular}{c|c}
\multicolumn{2}{c}{Cascadeloos}\\
$T_1$&$T_2$\\
\hline
read(A) &\\
& read(A)\\
&write(A)\\
&commit\\
write(A)&\\
read(B)&\\
write(B)&\\
commit&
\end{tabular}
\begin{tabular}{c|c}
\multicolumn{2}{c}{Niet cascadeloos}\\
$T_1$&$T_2$\\
\hline
read(A) &\\
write(A)&\\
& read(A)\\
read(B)&\\
&write(A)\\
write(B)&\\
commit&\\
&commit\\
\end{tabular}
\end{figure}
\begin{ei}
Elk cascadeloos transactierooster is herstelbaar.
\end{ei}

\subsection{Strictheid}
\begin{de}
Een rooster $S$ is \textbf{strict} wanneer elke transactie $T$ enkel waarden leest en/of schrijft na de commit van de laatste transactie die dat item geschreven heeft.
\end{de}
\begin{figure}[H]
\centering
\begin{tabular}{c|c}
\multicolumn{2}{c}{Niet strict}\\
$T_1$&$T_2$\\
\hline
read(A) &\\
& read(A)\\
&write(A)\\
write(A)&\\
read(B)&\\
write(B)&\\
&commit\\
commit&
\end{tabular}
\begin{tabular}{c|c}
\multicolumn{2}{c}{Strict}\\
$T_1$&$T_2$\\
\hline
& read(A)\\
read(B)&\\
write(B)&\\
& write(A)\\
&commit\\
read(A)&\\
write(A)&\\
commit&
\end{tabular}
\end{figure}
\begin{ei}
Elk strict transactierooster is cascadeloos (en dus ook herstelbaar).
\end{ei}

\subsection{Equivalentie}
\begin{de}
Twee roosters zijn \textbf{resultaat-equivalent} als ze hetzelfde resultaat geven bij een bepaalde input.
\end{de}
\begin{de}
Twee roosters zijn \textbf{conflict-equivalent} als de volgorde van elke twee conflicterende operaties dezelfde is in beide roosters.
\end{de}
\section{Serialisatie}
\begin{de}
Een rooster is \textbf{serieel} is een rooster waarin elke transactie afgemaakt wordt voor er aan een volgende wordt begonnen.
\end{de}
\begin{de}
Een rooster is \textbf{conflict-serialiseerbaar} als en slechts als het conflict-equivalent is met een serieel rooster met dezelfde transacties.
\end{de}
We kunnen serialiseerbaarheid testen, maar dit geeft nog steeds problemen.
\begin{itemize}
\item De manier waarop de operaties worden opgevolgd wordt bepaald door het besturingssysteem en is dus niet te voorspellen.
\item Transacties worden zo aangeboden dat het begin en einde vaak moeilijk te voorspellen is.
\item Als het rooster niet serialiseerbaar blijkt te zijn is herstel duur.
\end{itemize}

\end{document}
