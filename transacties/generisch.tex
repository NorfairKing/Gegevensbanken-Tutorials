\documentclass[transacties.tex]{subfiles}
\begin{document}

\chapter{Generisch}
\renewcommand\thesection{V\arabic{section}}
\renewcommand\thesubsection{V\arabic{section}}


\section{Is dit transactierooster herstelbaar?}
\subsection*{Opgave}
Gegeven een transactierooster $S$ van $n$ transacties $T_i$. Bepaal of $S$ herstelbaar is. (Als $S$ herstelbaar is, kijk dan na of het cascadeloos en strict is.)
\subsection*{Antwoord}
\begin{enumerate}
\item Bepaal voor elke variabele welke transacties ernaar schrijven.
\item Bepaal voor elke variabele welke transacties ervan lezen.
\item Check voor elke koppel $(T_1,T_2)$ transacties de noodzakelijke voorwaarde uit de definitie van herstelbaarheid.
\item Als alle checks slagen, is het transactierooster herstelbaar.
\end{enumerate}

\section{Is dit transactierooster conflict-serialiseerbaar?}
\subsection*{Opgave}
Gegeven een transactierooster $S$ van $n$ transacties $T_i$. Bepaal of $S$ conflict-serialiseerbaar is.
\subsection*{Antwoord}
We zullen een gerichte graaf opstellen aan de hand van $S$. Het rooster is conflict-serialiseerbaar als en slechts als deze graaf geen cycli bevat.
\begin{enumerate}
\item Maak een knoop $v_i$ voor elke transactie $T_i$.
\item Maak een book $e_{ij}$ voor elk van de volgende mogelijke gevallen.
\begin{itemize}
\item $T_i$ voert een \textbf{lees} operatie uit op $A$ \textbf{na} een \textbf{schrijf} operatie van $T_j$.
\item $T_i$ voert een \textbf{schrijf} operatie uit op $A$ \textbf{na} een \textbf{lees} operatie van $T_j$.
\item $T_i$ voert een \textbf{schrijf} operatie uit op $A$ \textbf{na} een \textbf{schrijf} operatie van $T_j$/
\end{itemize}
\end{enumerate}

% Examenvraag
\section{Bespreek dit transactierooster}
\subsection*{Opgave}
Bepaal of dit rooster herstelbaar, cascadeloos, strict, conflict-serialiseerbaar is.

\subsection*{Antwoord}
Gebruik bovenstaande vragen als handleiding om elk van de vragen individueel op te lossen.

% Examenvraag
\section{Timestamp ordering}
\subsection*{Opgave}
Gegeven een timestamp ordering algoritme.
\begin{itemize}
\item Leg uit wat $TS(T)$, $read\_TS(X)$ en $write\_TS(X)$ betekent.
\item Geef twee nadelen van standaard `Timestamp ordering'
\item Gegeven twee transacties, geef een mogelijk rooster met timestamp ordering.
\end{itemize}
\subsection*{Antwoord}


\end{document}
